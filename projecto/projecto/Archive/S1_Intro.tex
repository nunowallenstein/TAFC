
%----------------------------------------------------------------------------------------
%	SECTION 1
%----------------------------------------------------------------------------------------

\section{Introduction}
One of the key factors that distinguish humans from other primates is that they are capable of having a bipedal walking/running gait. This aspect served as an evolutionary remark towards our movement because humans could have a much higher mobility aswell as having a bigger vision range than before against their preys and it's predators. This along with the development of technology allowed for humans to have a bigger advantage against natural conditions and therefore contributed for our survival as species. Since the way we walk is unique and very complex \cite{Mochon&McMahon} regarding the rest the animal kingdom, a push has been made to reproduce the same type of walking and running in robots and other types of machines.

%Referencia cavagna
%Cavagna, G. A., Thys, H. & Zamboni, A. 1976 The sources
%of external work in level walking and running.

To accomplish the goal of imitating the same pattern of human walking and running, \cite{Seyfarth2006} describes that in the seventeenth century, Giovanni Afonso Borelli wrote ``De motu Animalum'', the first treatise on biomechanics, where he compared walking with vaulting over stiff legs although, while for running, he declared that the rebounding effect on compliant legs was important \cite{Borelli1685}. In an attempt to reproduce the same effects of human walking and running two models were created with entirely different dynamics. Walking was developed by \cite{Alexander1976,Mochon&McMahon,Cavagna1976} as a inverted pendulum model, while running was devolped by \cite{Blickhan1989,McMahon&Cheng1990}, as a spring-mass model. With these models, several properties could be obtained such as the kinetic and potential energies, the speed and frequency aswell as other aspects for the associated gait. The spring-mass model was able to explain the basics of running, although, regarding walking , it was verified this type of gaiting could not be imitated by using an inverted pendulum model \cite{Full&Koditchek1999}. One of the differences between the model and the experimental data points, for example, was the M shaped ground reaction force that is not present in the inverted pendulum model since  delta functions appear at the beggining and ending of one step \cite{Pandy2003}.

With this in account, \cite{Seyfarth2006} proposes a two compliant leg model consisting of two springs attached to a mass, being able to reproduce the same effects of walking with a relatively simple description despite the large complexity of the movement. In this paper, the stable parameter map is determined in relation with the variation of physical parameters such as the spring stiffness $k$. Also, it is shown that walking and running can be encapsulated into the same model with the energy associated to the system being the mediator between one type and the other type of gaiting.  After the success, \cite{Rummel2010} uses the same model to determine 5 types of walking gaits, aswell as to test the robustness of the model by determining the regions of attraction along with the saddle region associated to the parameter space.






